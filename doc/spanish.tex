\selectlanguage{spanish}
\chapter*{Español}

\section*{Instalación}

Descarga el juego de la última publicación del repositorio oficial\footnote{%
	\url{https://github.com/Groctel/cgpn-sg}
}
o clónalo para jugar a la última versión:

\begin{lstlisting}[language=sh]
git clone https://github.com/Groctel/cgpn-sg
\end{lstlisting}

Necesitas instalar las dependencias del juego con \texttt{npm} o \texttt{yarn} antes de ejecutar CGPN\@.
Cualquiera de estas órdenes instalarán las dependencias localmente en un directorio llamado \texttt{node\_modules}.
Ejecútalas desde el directorio superior del juego (donde está el fichero \texttt{README.md}):

\begin{lstlisting}[language=sh]
npm install
yarn install
\end{lstlisting}

Ejecuta el juego con la misma orden que usaste en el paso anterior sustituyendo \texttt{install} por \texttt{start}:

\begin{lstlisting}[language=sh]
npm start
yarn start
\end{lstlisting}

Esto debería abrir automáticamente tu navegador y empezar a ejecutar el juego.
Si no lo hace, ábrelo y ve a \texttt{localhost:8080} para empezar a jugar.
Recarga la pestaña para reiniciar el juego.
Para parar el juego, pulsa \texttt{\^{}C} en la terminal donde se está ejecutando el servidor del juego.

\section*{Controles}

Este juego usa los controles clásicos de \texttt{WASD} y ratón en primera persona.
Usa tu ratón para rotar la cámara desde el punto de vista del jugador.
Usa las siguientes teclas para mover al jugador:

\begin{itemize}
	\item
		\texttt{A}\textbf{:}
		Mover el jugador a la izquierda.
	\item
		\texttt{D}\textbf{:}
		Mover el jugador a la derecha.
	\item
		\texttt{S}\textbf{:}
		Mover el jugador hacia delante.
	\item
		\texttt{W}\textbf{:}
		Mover el jugador hacia atrás.
	\item
		\texttt{ESPACIO}\textbf{:}
		Saltar.
\end{itemize}

Si es la primera vez que juegas con este esquema de controles coloca el anular de tu mano izquierda sobre la \texttt{A}, tu dedo corazón sobre la \texttt{W} y tu índice sobre la \texttt{D}.
Esto te permitirá pulsar la \texttt{S} moviendo el dedo corazón hacia ti desde la \texttt{W} y alcanzar fácilmente el \texttt{ESPACIO} con tu pulgar.

Usa tu razón para alterar el mundo a tu gusto:

\begin{itemize}
	\item
		\texttt{CLICK IZQUIERDO}\textbf{:}
		Eliminar un cubo.
	\item
		\texttt{CLICK DERECHO}\textbf{:}
		Colocar un cubo.
	\item
		\texttt{RUEDA DEL RATÓN}\textbf{:}
		Cambiar el cubo sostenido por el jugador.
\end{itemize}

Los cubos serán destruidos y colocados donde está mirando el jugador, guiado por la cruceta en el centro de la pantalla.
¡Si estás muy lejos de un cubo tendrás que acercarte para ineractuar con él!
